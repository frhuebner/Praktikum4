\subsection{Durchführung}
Für den Frank-Hertz-Versuch wird zunächst der Ofen auf einne Temperatur von etwa $165^\circ$C geheizt. Mit dem Computer wird die Spannung $U_\mathrm{A}$ (proportional zu $I_\mathrm{A}$) und die Beschleunigungsspannung $U_\mathrm{B}$ aufgenommen. Der Aufbau erlaubt ein automatisches Hochfahren der Beschleunigunsspannung bis zu einem eingestellten Maximalwert. Bei der eingestellten Temperatur von $T=165^\circ$C wird für vier verschiedene Gegenspannungen die Kennlinie aufgenommen. Anschließend wird die Kennlinie bei fester Gegenspannung für vier verschiedene Temperaturen aufgenommen.
