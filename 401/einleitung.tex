\section{Einleitung}
In diesem Versuch werden elektrische Übergänge in Atomen untersucht. Zunächst wird dabei der Zeemann-Effekt untersucht, dem die Aufspaltung der Energieniveaus durch ein äußeres Magnetfeld zugrundeliegt. Die Aufspaltung der Energieniveaus kann durch die Beobachtung der emittierten Photonen durch ein Interferometer analysiert werden. Der Frank-Hertz-Versuch stellt den zweiten Versuchsteil dar. Hierbei werden Elektronen durch Quecksilber beschleunigt und das Stoßverhalten untersucht. Hieraus lässt sich auf die Anregungsenergie des beteiligten Übergangs von Quecksilber schließen.
