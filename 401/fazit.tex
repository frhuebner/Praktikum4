\section{Fazit}
Wir haben Bilder von der Oberfläche von Graphit, Gold und TaS$_2$ in der Größenordung von 100nm aufgenommen. Zusätzlich war es möglich ein Bild der Graphitoberfläche mit atomarer Auflösung zu erstellen. Mithilfe dieser Aufnahme konnten wir nach Entzerrung des Bildes bestätigen, dass die Bindungswinkel der Graphitstruktur $120^\circ \pm 3^\circ$ beträgt, es sich also um eine Struktur aus regelmäßigen Sechsecken handelt. Der Atomabstand wurde zu $a = 0,179\si{\nano\metre} \pm 0,002\si{\nano\metre}$ bestimmt. Dies weicht deutlich vom Literaturwert $l=0,142$ nm ab. Gründe für diese Abweichung wurden erläutert.



