\section{Fazit}
Die Aufspaltung der Interferenzringe beim Zeeman-Effekt konnte gut beobachtet werden. Der Zusammenhang zwischen der Polarisation des Lichtes und der Richtung des Magnetfeldes wurde bestätigt. Mithilfe der aufgenommenen Daten wurde das Bohrsche Magneton bestimmt. Der Literaturwert liegt innerhalb der Fehlergrenzen des von uns bestimmten Wertes. Das ermittelte Auflösungsvermögen und die Finesse des Etalons stimmen bis auf einen Fall mit den theoretischen Werten überein. Gründe für die Abweichung bei einem Wert wurden erläutert. \\ 
Die aufgenommenen Kennlinien beim Franck-Hertz-Versuch zeigen den erwarteten Verlauf. Aus den Kennlinien wurde die Anregungsenergie des beobachteten Übergangs berechnet. Diese stimmt genau mit dem Literaturwert überein. Das Verhalten der Kennlinie bei Änderung von Temperatur und Gegenspannung zeigt ebenfalls den erwarteten Verlauf und bestätigt, dass die experimentelle Durchführung auf ein kleines Temperaturintervall beschränkt ist.
