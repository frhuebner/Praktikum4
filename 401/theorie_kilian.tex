\subsection{Theoretische Grundlagen}
Mit dem Franck-Hertz-Versuch kann auf die Energieabstände in den untersuchten Atomen geschlossen werden. Der Aufbau ist in Abbildung \ref{fig:frank} zu sehen. Durch die Heizspannung $U_\mathrm{H}$ werden Elektronen aus der Glühkathode gelöst. Diese werden dann durch die Beschleunigungsspannung $U_\mathrm{B}$ in Richtung des Gitters beschleunigt. Auf dem Weg zum Gitter können sie mit Atomen elastisch und inelastisch stoßen. Wenn die Elektronen am Gitter noch schnell genug sind, um die Gegenspannung $U_\mathrm{G}$ zu überwinden, können sie die Anode erreichen und so zum Strom $I_\mathrm{A}$ beitragen. Im Folgenden wird vereinfacht davon ausgegangen, dass für die Atome nur zwei Energiezustände mit Energiedifferenz $\Delta E$ möglich sind. Ein erstes Minimum des Anodenstroms erhält man dann, wenn die Elektronen die kinetische Energie $\Delta E$ haben und so die Atome anregen können (inelastischer Stoß). Das zweite Minimum folgt, wenn die Elektronen nach einem Stoß wieder schnell genug werden können, um erneut inelastisch zu stoßen. Insgesamt folgt so eine periodische Abfolge von Minima und Maxima des Anodenstroms. Da aus $U_\mathrm{B}$ die Energie, die ein Elektron pro Durchlauf aufnimmt, berechnet werden kann, kann man aus der Periodizität der $I_\mathrm{A}$-$U_\mathrm{B}$-Kennlinie auf die Anregungsenergie $\Delta E$ schließen.

\begin{figure}[h]
  \centering
  \begin{tikzpicture}
    \draw (0,5)--(6,5);
    \draw (0,2)--(6,2);
    \draw (0,5) arc (90:270:1.5);
    \draw (6,2) arc (-90:90:1.5);
    \draw [dotted, thick] (4,2)--(4,5);
    \draw (4,5.5) node {Gitter};
    \draw (7,2.5)--(7,4.5);
    \draw (7,5.5) node {Anode};
    \draw [spring] (-1,2.5)--(-1,4.5);
    \draw [only marks, samples=100, mark size=0.75, mark=*,domain=-0.3:4] plot(\x,{3.5+0.45*rand*3});
    \draw [->] (-0.8,3.5)--(-0.4,3.5);
    \draw (-0.6,3.7) node {e$^-$};
    \draw (-1,4.5)--(-2,4.5);
    \draw (-2,4.5)--(-2,3.7);
    \draw (-1,2.5)--(-2,2.5);
    \draw (-2,2.5)--(-2,3.3);
    \draw (-2.1,3.3)--(-1.9,3.3);
    \draw (-2.2,3.7)--(-1.8,3.7);
    \draw (-2.8,3.5) node {$U_\mathrm{H}$};
    \draw (-2.3,3.2) node {-};
    \draw (-2.3,3.8) node {+};
    \draw (-1.5,2.5)--(-1.5,1.5)--(1,1.5);
    \draw (1.6,1.5)--(4,1.5)--(4,2);
    \draw (4,1.5)--(5,1.5);
    \draw (5.6,1.5)--(7,1.5);
    \draw (7.25,1.5) circle (0.25);
    \draw [->] (7,1.25)--(7.5,1.75);
    \draw (7.25,0.8) node {$I_\mathrm{A}$};
    \draw (7.5,1.5)--(8,1.5)--(8,3.5)--(7,3.5);
    \draw (5,1.3)--(5,1.7);
    \draw (5.6,1.4)--(5.6,1.6);
    \draw (5.3,0.8) node {$U_\mathrm{G}$};
    \draw (4.8,1.3) node {+};
    \draw (5.8,1.3) node {-};
    \draw (1.6,1.3)--(1.6,1.7);
    \draw (1,1.4)--(1,1.6);
    \draw (1.3,0.8) node {$U_\mathrm{B}$};
    \draw (0.8,1.3) node {-};
    \draw (1.8,1.3) node {+};
    \draw (-1,5.5) node {Glühkathode};
    \draw [->] (-0.8+6,3.5)--(-0.4+6,3.5);
    \draw (-0.6+6,3.7) node {e$^-$};
  \end{tikzpicture}
  \caption{Gasentladungsröhre mit Quecksilber}
  \label{fig:frank}
\end{figure}
