\subsection{Durchführung}

\begin{figure}[h]
  \centering
  \begin{tikzpicture}
    \draw (-1.5,0) circle (0.3);
    \draw (-0.21-1.5,-0.21) -- (0.21-1.5,0.21);
    \draw (-0.21-1.5,0.21) -- (0.21-1.5,-0.21);
    \draw (4,-0.5) -- (4,-0.1);
    \draw (4,0.5) -- (4,0.1);
    \draw (1.5,0) arc (0:30:1);
    \draw (1.5,0) arc (0:-30:1);
    \draw (1.24,0) arc (180:150:1);
    \draw (1.24,0) arc (180:210:1);
    \draw (6.5,0) arc (0:30:1);
    \draw (6.5,0) arc (0:-30:1);
    \draw (6.24,0) arc (180:150:1);
    \draw (6.24,0) arc (180:210:1);
    \draw [pattern=north west lines, rotate around={-45:(10,0)}] (9.9,0.5) rectangle (10.1,-0.5);
    \draw [rotate around={-45:(10,0)}] (7.5,0) arc (0:30:1);
    \draw [rotate around={-45:(10,0)}] (7.5,0) arc (0:-30:1);
    \draw [rotate around={-45:(10,0)}] (7.24,0) arc (180:150:1);
    \draw [rotate around={-45:(10,0)}] (7.24,0) arc (180:210:1);
    \draw [rotate around={-45:(10,0)}] (4.9,0.5) rectangle (5.1,-0.5);
    \draw (-1.5,-1) node {Hg-/Balmer-};
    \draw (-1.5,-1.4) node {Lampe};
    \draw (1.4,-1) node {Linse};
    \draw (1.4,-1.4) node {$f=50 \ \mathrm{mm}$};
    \draw (4,-1) node {Spalt};
    \draw (6.4,-1) node {Objektiv};
    \draw (6.4,-1.4) node {$f=150 \ \mathrm{mm}$};
    \draw (10,-1) node {Gitter};
    \draw [thick, dash dot] (-1.2,0) -- (1.25,0);
    \draw [thick, dash dot] (1.5,0) -- (6.2,0);
    \draw [thick, dash dot] (6.5,0) -- (10,0);
    \draw [thick, dash dot] (10,0) -- (8.25,1.75);
    \draw [thick, dash dot] (8,2) -- (6.6,3.4);
    \draw (10,2.5) node {Linse};
    \draw (10,2.1) node {$f=300 \ \mathrm{mm}$};
    \draw (8.6,4) node {Okular/Kamera};
    \draw [thick, dotted] (10,0) -- (13,0);
    \draw [thick, dotted] (10,0) -- (12,-2);
    \draw (12,0) arc (0:-45:2);
    \draw (11.5,0) arc (0:135:1.5);
    \draw (11.2,-0.6) node {$\omega_\mathrm{G}$};
    \draw (10.6,0.8) node {$\omega_\mathrm{B}$};
  \end{tikzpicture}
  \caption{Aufbau zur Untersuchung der Balmer-Serie}
  \label{aufbau2}
\end{figure}

Bevor man die Balmerlinien des Wasserstoffs vermessen kann, muss man zuerst die Gitterkonstante des Reflexionsgitters bestimmen. Dazu werden die bekannten Spektrallinien einer Hg-Lampe gemessen. Um eine Spektrallinie zu vermessen dreht man das Gitter, bis die Linie mittig im Okular erscheint und misst dann die beiden Winkel $\omega\ind{G}$ und $\omega\ind{B}$. Mit diesen Werten und der Wellenlänge der Linie wird dann die Gitterkonstante bestimmt (siehe Kap. \ref{subsec:auswertung_spektrallinien}).\\

Danach wird die Messung für die Spektrallinien der Balmer-Lampe wiederholt. Aus diesen Daten kann man nun mithilfe der Gitterkonstanten die Wellenlänge der Linien bestimmen. Um auch die Isotopieaufspaltung zu messen wird der Abstand der beiden Linien mit der Skala im Okular bestimmt.\\

Weiterhin wird die Isotopieaufspaltung mit einer CCD-Kamera vermessen, die die Intensität in Abhängigkeit des Winkels misst. Man dreht das Gitter wieder, bis die Linie etwa mittig auf der Kamera erscheint. Um das Rauschen etwas zu verringern werden die Daten über ein paar Sekunden gemittelt.