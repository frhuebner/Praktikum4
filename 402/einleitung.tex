\section{Einleitung}
Die Erklärung des Photoeffekts hat wesentlich zur Entwicklung der Quantenmechanik beigetragen.
In diesem Versuch soll deshalb genau dieser untersucht werden. Ziel ist es dabei das Plancksche Wirkungsquantum und die Austrittsarbeit der Anode zu bestimmen. Anschließend wird mit Hilfe eines Reflexionsgitters die Balmer-Serie von Wasserstoff untersucht und unter anderem die Rydberg-Konstante und erneut das Plancksche Wirkungsquantum bestimmt. Dabei wird auch die Isotopieaufspaltung untersucht, die ebenfalls quantenmechanisch erklärt werden kann.
