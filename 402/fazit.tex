\section{Fazit}
Die gemessenen Kennlinien des Photoeffektes zeigen den erwarteten Verlauf. Der berechnete Wert für das plancksche Wirkungsquantum weicht um ca. $8\%$ vom echten Wert ab, welcher nicht innerhalb der Fehlergrenzen liegt. Eine mögliche Ursache dafür und für das Abweichen der Austrittsarbeit um etwa $70\%$ wurde erläutert. Auch das Verhalten des Photostroms bei veränderter Intensität entspricht den Erwartungen. \\

Die mit dem Reflexionsgitter bestimmten Wellenlängen der Balmerlinien entsprechen (innerhalb der Fehlergrenzen) den theoretischen Wellenlängen. Die berechnete Rydbergkonstante weicht um ca. $3\%$ ab, der Literaturwert liegt aber knapp außerhalb der Fehlergrenzen. Gleiches gilt für das bestimmte Wirkungsquantum welches ebenfalls um etwa $3\%$ abweicht. Für zwei Emissionslinien konnte außerdem die Isotopieaufspaltung sowohl mit dem Okular als auch mit der CCD-Kamera beobachtet werden. 
