\subsection{Theoretische Vorbetrachtungen}
\subsubsection{Spektrallinien}
Die Energie-Eigenzustände des Wasserstoffatoms werden durch die Quantenzahlen $n,l$ und $m$ beschrieben. Ein Zustand mit der Hauptquantenzahl $n$ hat dabei den Energie-Eigenwert $E_n=-R\ind{y}/n^2$, wobei $R\ind{y} = hc\cdot \cfrac{R_\infty}{1+\frac{m\ind{e}}{m\ind{K}}}$ die Rydberg-Energie ist ($m\ind{K}$ Kernmasse). Wechselt das Wasserstoffatom von einem Zustand mit Hauptquantenzahl $n$ zu einem energetisch niedrigeren Zustand mit Hauptquantenzahl $n'$ wird ein Photon der Energie 
\begin{align*}
  E_{n \rightarrow n'}&=R\ind{y} \left( \frac{1}{n'^2} - \frac{1}{n^2}\right)
\end{align*}
emittiert. Die Balmer-Serie enthält alle Frequenzen des Emissionsspektrums, die aus dem Übergang von einem Niveau $n>2$ auf das Niveau $n'=2$ entstehen. Daraus ergeben sich folgender Zusammenhang für die Wellenlängen:
\begin{align}
  \cfrac{1}{\lambda} = \cfrac{R_\infty}{1+\frac{m\ind{e}}{m\ind{K}}} \left( \frac{1}{2^2} - \frac{1}{n^2}\right)
  \label{equ:rydberg}
\end{align}

Daran kann man auch die Isotopieaufspaltung erkennen: Für unterschiedliche Kernmassen $m\ind{K}$ ergeben sich leicht unterschiedliche Wellenlängen. 

\subsubsection{Linienbreite}
In der Praxis haben die beobachtbaren Balmer-Linien eine gewisse Ausdehnung, mögliche Ursachen dafür sind:
\begin{itemize}
\item    
Durch die thermische Bewegung der Atome kommt es zur Dopplerverschiebung der Wellenlänge des emittierten Lichtes. Bewegt sich das Atom mit der Geschwindigkeit $v$ auf den Empfänger zu und emittiert ein Photon der Wellenlänge $\lambda$, so misst dieser die Wellenlänge 
\begin{align*}
  \lambda'=\lambda \cdot \left(  1+ \frac{v}{c} \right),
\end{align*} 
wobei $c$ die Lichtgeschwindigkeit ist. Nimmt man an, dass es sich bei dem Gas der Gasentladungsröhre um ein ideales Gas handelt, gilt für die Geschwindigkeit $v$ jedes Atoms (mit Masse $m$ und Boltzmann-Konstante $k_\mathrm{B}$)
\begin{align*}
  \frac{1}{2} m\ind{K} v^2=\frac{3}{2} k_\mathrm{B} T.
\end{align*}
Durch die Dopplerverschiebung schwankt die Wellenlänge also im Bereich
\begin{align}
  \sigma[\lambda]\ind{Doppler}=\frac{\lambda}{c}\sqrt{\frac{3k\ind{B}T}{m\ind{K}}}.
  \label{equ:doppler}
\end{align}
\item Es gibt eine natürliche Linienbreite. Nach \cite{unschaerfe} gilt, dass wenn ein angeregter Zustand des Atoms die mittlere Lebensdauer $\tau$ hat, die Spektrallinie dieses Zustandes die Form einer Lorentz-Kurve mit Breite 
\begin{align}
  \sigma[E]\ind{n}&=\frac{\hbar}{\tau}\nonumber\\
  \sigma[\lambda]\ind{n} &= \frac{\lambda^2}{2\pi c \tau}
  \label{equ:natbreite}
\end{align}
hat.
\end{itemize}