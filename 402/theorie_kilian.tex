\subsection{Theoretische Vorbetrachtungen}
Beim Photoeffekt treffen Photonen auf eine Photozelle. In der Photozelle befinden sich (getrennt voneinander) eine Anode und eine Kathode. Diese sind an eine äußere Spannung $U_\mathrm{G}>0$ angeschlossen. Betrachtet man die Elektronen im Bändermodell, so führt die Verbindung mit der Spannungsquelle dazu, dass die Fermie-Niveaus von Anode ($E_\mathrm{FA}$) und Kathode ($E_\mathrm{FK}$) nun eine Potentialdifferenz von $e U_\mathrm{G}$ haben, wobei $e$ die Elementarladung ist. Bei unterschiedlichen Austrittsarbeiten $W_\mathrm{A}$ und $W_\mathrm{K}$ von Anode und Kathode entsteht eine Potentialdifferenz von $eU_\mathrm{AK}=W_\mathrm{A}-W_\mathrm{K}$, das so genannte Kontaktpotential (siehe Abbildung \ref{Photozelle}). 

\begin{figure}[h]
  \centering
  \begin{tikzpicture}
    \draw (0,0)--(2,0);
    \draw (-0.4,0) node {$E_\mathrm{FK}$};
    \draw (2,1)--(4,1);
    \draw (4.4,1) node {$E_\mathrm{FA}$};
    \draw [<->] (2,0.1)--(2,0.9);
    \draw (2.5,0.5) node {$eU_\mathrm{G}$};
    \draw [<->] (0.3,0.1)--(0.3,1.9);
    \draw (-0.1,1) node {$W_\mathrm{K}$};
    \draw [<->] (3.7,1.1)--(3.7,3.9);
    \draw (4,2.5) node {$W_\mathrm{A}$};
    \draw [thick, dash dot] (0,2)--(2,2);
    \draw [thick, dash dot] (2,4)--(4,4);
    \draw [<->] (2,2.1) --(2,3.9);
    \draw (1.4,3) node {$eU_\mathrm{KA}$};
    \draw (-0.5,-0.3)--(-0.5,-1);
    \draw (-0.5,-1)--(1.5,-1);
    \draw (4.3,0.7)--(4.3,-1);
    \draw (4.3,-1)--(2.5,-1);
    \draw (1.55,-1) circle (0.05cm);
    \draw (1.55,-1.3) node {$+$};
    \draw (2.45,-1) circle (0.05cm);
    \draw (2.45,-1.3) node {$-$};
    \draw (2,-1) node {$U_\mathrm{G}$};
  \end{tikzpicture}
  \caption{Energieniveaus in Photozelle}
  \label{Photozelle}
\end{figure}

Damit nun ein gebundenes Elektron der Kathode zu der Anode gelangen kann muss es die Potentialdifferenz $W_\mathrm{A}+eU_\mathrm{G}$ überwinden. Ein einzelnes Photon hat eine von der Frequenz $\nu$ abhängige Energie von $E=h\nu$, wobei $h$ das Plancksche Wirkungsquantum ist. Reicht diese Energie aus, um die Potentialdifferenz zwischen Kathode und Anode zu überwinden, so kann ein Strom zwischen den Elektroden fließen. Bei der Grenzspannung $U_0$ ist die Energie der Photonen gerade so groß, dass ein kleiner Strom fließen kann. Dann folgt die Energiebilanz
\begin{align}
  h\nu=eU_0+W_\mathrm{A}.
  \label{eqn:grenzspannung}	
\end{align}
Hierbei wird angenommen, dass $W_\mathrm{A}+eU_\mathrm{G} \geq W_\mathrm{K}$ gilt. Ist dies nicht der Fall, müsste lediglich die Potentialdifferenz $W_\mathrm{K}$ überwunden werden und $W_\mathrm{K}$ könnte bestimmt werden. \\ \\
Aus \cite{kennlinie} folgt (mit $h\nu>eU_\mathrm{G}+w_\mathrm{A}$), dass bei kleinen Temperaturen für die Zahl der herausgelösten Elektronen $N$ und den Photostrom $I$
\begin{align}
  I \propto N \propto \left(  h\nu-W_\mathrm{A}-eU_\mathrm{G}\right)^2
\end{align} 
gilt. 
