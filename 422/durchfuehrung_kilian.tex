\section{Durchführung}
Der Aufbau besteht aus dem Rastertunnelmikroskop und einem Computer an dem Einstellungen vorgenommen und die Bilder betrachtet werden können. Die Probe die beobachtet werden soll wird auf einem magnetischen Halter platziert und der Halter vor die Spitze gelegt. Über den Computer kann dann manuell die Probe weiter an die Spitze herangefahren werden. In das Mikroskop ist eine Lupe integriert um die Probe so schon relativ nah an die Spitze zu bringen. Anschliessen wird die automatische Annäherung gestartet. Dies dauert einige Minuten (abhängig davon wie nah die Probe manuell platziert wurde).\\
Anschliessend kann die Messung gestartet werden. Während der Messung wird die Probe immer wieder abgerastert und das Bild angezeigt und gespeichert. Die für uns wichtigen Einstellungen sind die drei Koeffizienten des PID-Reglers (hier $P$, $I$ und $D$) genannt. Für die CCM wählen wir $P=1000$, $I=2000$ und $D=0$ und für die CHM $P=0$, $I=4$ und $D=0$. Eine weitere Einstellung die wir nutzen ist die Zeit $\tau$, die die Spitze braucht um eine Linie abzurastern. Für die CCM (CHM) wählen wir $\tau=0,2$s ($\tau=0,5$s).\\
Bei allen Proben beginnen wir mit großen Bildausschnitten und wählen langsam immer größere Vergrößerungen. Dabei wird fast ausschließlich mit der CCM angefangen und das endgültige Bild aber mit der CHM aufgenommen, da die Strukturen mit dieser besser zu erkennen sind. Es fällt auf, wie wichtig es ist, bei den Messungen keine Erschütterungen zu verursachen. Sogar kleinste Bewegungen in der Nähe des Tisches werden auf den Bildern als Verwischung sichtbar.\\
Zunächst werden zwei Proben aus Graphit und Gold untersucht. Anschließend wird noch versucht in TaS$_2$  Ladungsdichtewellen zu beobachten. 
