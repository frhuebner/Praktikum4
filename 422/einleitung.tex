\section{Einleitung}
Mit dem Rastertunnelmikroskop können unter Ausnutzung des quantenmechanischen Tunneleffektes elektrisch leitende Proben untersucht werden. Die Auflösung, die dabei erreicht werden kann, übertrifft die Auflösung von gewöhnlichen Mikroskopen deutlich.\\
In diesem Versuch werden Proben aus Graphit, Gold und TaS$_2$ untersucht. Dabei werden verschiedene Betriebsmodi verwendet. Die beobachtbare Kristallstruktur von Graphit wird näher untersucht und mit der erwarteten Struktur verglichen.  
