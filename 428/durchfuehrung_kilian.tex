\section{Durchführung}

\subsection{Bragg-Reflexion}
Für die Bragg-Reflexion wird das Zählrohr in das Vollschutzröntgengerät eingebaut. Dann wird die unbekannte Röntgenröhre eingebaut. Auf den Targethalter wird der NaCl-Kristall für die Beugung der Röntgenstrahlung platziert. Mit den in \cite{praktikumsheft} angegebenen Einstellungen wird nun der Messvorgang am PC gestartet. Das Zählrohr fährt Winkel zwischen $\beta_\mathrm{min}=2^\circ$ und $\beta_\mathrm{max}=25^\circ$ ab während der Targethalter um den halben Winkel mitgedreht wird. \\
Für die Nachtmessung wird die Molybdän-Anode eingesetzt. Mit den Einstellungen aus \cite{praktikumsheft} wird zwischen $28,5^\circ$ und $32^\circ$ gemessen.

\subsection{Messung für Materialanalyse}
Für die Materialanalyse werden der Röntgenenergiedetektor und die Kupfer-Anode verwendet. Auf dem Targethalter werden nacheinander das Kalibrierungstarget (FeZn), vier unbekannte Proben und zwölf weitere Proben mit bekannten Bestandteilen platziert. Das Target steht jeweils im Winkel von $45^\circ$ und der Detektor im Winkel von $90^\circ$ zum Kollimator. Mit den in \cite{praktikumsheft} angegebenen Einstellungen wird für jede Probe 180 Sekunden lang gemessen. 
