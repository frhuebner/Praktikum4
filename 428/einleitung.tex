\section{Einleitung}
In diesem Versuch werden die Entstehung und die Nutzung von Röntgenstrahlung untersucht. Dazu wird zunächst über die Bragg-Reflexion das charakteristische Spektrum einer unbekannten Röntgenanode aufgenommen und so auf das Gas in der Anode geschlossen. \\
In einer Nachtmessung wird eine Aufnahme der Feinstrukturaufspaltung einer Molybdän-Emissionslinie erstellt. \\
Für verschiedene Proben wird das Emissionsspektrum mit Hilfe eines Röntgenenergiedetektors aufgenommen. Mit den aufgenommenen Referenzspektren können dann die Massenanteile der verschiedenen Elemente in den unbekannten Proben ermittelt werden. Mit dieser Methode können also chemische Zusammensetzungen von Materialien ohne deren Zerstörung ermittelt werden. \\
Im letzen Versuchsteil wird mit einer Laue-Aufnahme die Gitterstruktur eines Kristalls untersucht. 
