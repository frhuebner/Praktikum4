\section{Fazit}
Wir konnten aus dem Spektrum der unbekannten Anode das Material als Wolfram identifizieren. In der Nachtmessung war die Feinstrukturaufspaltung von Molybdän zu erkennen. Der Wellenlängenabstand wurde zu $\Delta \lambda = (0,428 \pm 0,0006) \si{\pico\meter}$ bestimmt.\\
Im zweiten Versuchsteil wurde versucht aus den Resonanzlinien der Stoffproben, die Zusammensetzung des Stoffes zu bestimmen. Dies gelang eindeutig nur für die erste Probe: Es handelt sich um ein Gemisch aus Eisen und Chrom. Bei der zweiten und dritten Probe vermuten wir Kupfer. Die vierte Probe konnte leider nicht bestimmt werden.\\
Im letzten Versuchsteil konnten wir eine Laue-Aufnahme eines NaCl-Kristalles entwickeln und fast allen Reflexen eindeutig ein Trippel von Miller-Indizes zuzuordnen. 