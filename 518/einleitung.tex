Bereits Anfang des 20. Jahrhunderts wurde die Höhenstrahlung entdeckt. Trotzdem kennt man weitgehend auch heute noch nicht ihre Ursachen. Mittlerweile wird mit großem Aufwand versucht, der Herkunft der Höhenstrahlung auf die Spur zu kommen. Ein wichtiger Aspekt ist dabei die Beobachtung der sekundären kosmischen Strahlung, welche durch die Wechselwirkung der ursprünglichen Strahlung mit der Atmosphäre entsteht. In diesem Versuch wird im ersten Teil die Winkelverteilung der Teilchen aus der sekundären Höhenstrahlung vermessen. Im zweiten Versuchsteil wird außerdem die Verteilung der Zerfallszeiten der kosmischen Myonen bestimmt.   
