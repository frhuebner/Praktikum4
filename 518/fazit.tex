Im Experiment war es möglich die Winkelverteilung zu messen. Trotz relativ großer Fehler konnte bestätigt werden, dass die Winkelverteilung durch eine $\cos^2{\varphi}$-Verteilung angenähert werden kann. Allerdings kommt es bei kleinen Winkeln zu großen Schwankungen. Die Verteilung des Energieverlustes der Teilchen in dem Detektor wurde als Landauverteilung verifiziert. Nur bei großen Energien hat die Verteilung aufgrund der Dicke des Detektors einen gaußförmigen Schwanz. Das Maximum der Verteilung liegt bei etwa 2,7 \si{\mega\eV}, was in der erwarteten Größenordnung liegt.\\

Im zweiten Versuchsteil wurde die mittlere Lebensdauer von Myonen auf $\tau = \SI[separate-uncertainty = true]{2.2(1)}{\micro\second}$ bestimmt. Das ist in Übereinstimmung mit dem Literaturwert \SI{2.197}{\micro\second}\cite{pdg}.
