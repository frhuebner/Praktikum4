Im Experiment war es uns möglich die Winkelverteilung zu messen. Trotz relativ großer Fehler konnten wir erkennen, dass man die Winkelverteilung durch eine $\cos^2{\phi}$-Verteilung annähern kann. Allerdings kommt es bei kleinen Winkeln zu großen Schwankungen. Wir haben auch die Verteilung des Energieverlustes der Teilchen in dem Detektor als Landauverteilung verifiziert können. Nur bei großen Energien hat die Verteilung aufgrund der Dicke des Detektors einen gaussförmigen Schwanz. Das Maximum der Verteilung liegt bei etwa 2,7 \si{\mega\eV}, was in der erwarteten Größenordnung liegt.\\
Im zweiten Versuchsteil haben wir die mittlere Lebensdauer von Myonen auf $\tau = \SI[separate-uncertainty = true]{2.2(1)}{\micro\second}$ bestimmt. Das ist in Übereinstimmung mit dem Literaturwert \SI{2.197}{\micro\second}\cite{pdg}.