\section{Fazit}
Das Signal zu den Rabi-Oszillationen zeigt den erwarteten Verlauf. Auch die aufgenommenen FID-Signale fallen wie erwartet exponentiell ab und bestätigen die zugrundel iegende Differentialgleichung.\\
Die longitudinale Relaxationszeit wurde mit beiden vorgestellten Methoden ermittelt. Die beiden Werte die wir dabei erhalten weichen allerdings um ca $14\%$ vom Mittelwert ab. Mögliche Gründe für diese Abweichung wurden angeführt.\\
Die effektive transversale Relaxationszeit konnte aus dem FID-Signal bestimmt werden. Der Vergleich mit den drei ermittelten transversalen Relaxationszeiten bestätigt, dass der Zerfall durch die Inhomogenität des Magnetfelds schneller vonstattengeht. Die drei bestimmten transversalen Relaxationszeiten weichen in einem Rahmen voneinander ab, der wegen der Verwendung verschiedener Pulssequenzen erwartet worden ist. \\ 
Ein Vergleich der ermittelten Zeitkonstanten mit Literaturwerten ist leider nicht möglich, da uns nichts Genaueres über die Mineralölprobe bekannt ist.          
