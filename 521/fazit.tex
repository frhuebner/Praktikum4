Die aufgenommenen Spektren zeigen qualitativ den erwarteten Verlauf.\\

 Die Energiekalibrierung wurde für beide Detektoren durchgeführt. Während bei dem Szintillationszähler nur die Hälfte der Datenpunkte auf der Kalibrationsgerade liegt, liegen beim Halbleiterdetektor alle Datenpunkte auf der Gerade und die Kalibration hat somit eine höhere Güte. \\

Für beide Detektoren und jeweils zwei Gammalinien wurde das Peak-to-Total Verhältnis bestimmt. Bei beiden Linien hat der Halbleiterdetektor das schlechtere Verhältnis, also eine geringere Nachweisgüte.\\

Die absolute Peakeffizienz wurde für beide Detektoren anhand des Cs-Spektrums bestimmt. Während der Halbleiterdetektor eine absolute Peakeffizienz von $(2,8 \pm 0,4) \%$ hat, ist sie beim Szintillationszähler mit $(9 \pm 2)\%$ etwa drei mal so groß.\\

Für den Halbleiterdetektor wurden zusätzlich noch die intrinsische Halbewertsbreite und die relative Effizienz berechnet. Die intrinsische Halbwertsbreite zeigt den erwarteten Verlauf. Die relative Effizienz sinkt mit der Energie und nähert sich einem konstanten Wert an.\\

In zwei Langzeitmessungen wurden die Spektren des Untergrundes und einer Bodenprobe aufgenommen. Die Bodenprobe zeigt keine zusätzlichen Gammalinien und wirkt sogar eher abschirmend. Deshalb wird das Spektrum des Untergrundes analysiert. In ihm konnten neben Gammalinien aus der Uran-Actinium- und der Uran-Radium-Reihe auch Emissionslinien von Kalium und Cäsium gefunden werden. Die Kaliumlinie entsteht durch das natürlich in der Umwelt vorkommende Kalium. Die Cäsium-Linie entsteht durch Cäsium aus der Reaktorkatastrophe in Tschernobyl. Der Cäsium-Peak hat die geringste Höhe der beobachteten Peaks. 
