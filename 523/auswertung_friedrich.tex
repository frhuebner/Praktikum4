\subsection{Untergrundmessung}

Die Untergrundmessung ergab $259\pm 16$ Ereignisse\footcite{Der Fehler beträgt $\sqrt{N}$} in \SI{200}{\second}. Das entspricht $N_U = 52 \pm 3$ Untergrundereignissen in \SI{40}{\second}.
 

\subsection{Thalliumprobe}
\begin{table}
\centering
\caption{Gemessene Ereignisse der Thalliumprobe}
\begin{tabular}{cc}
\toprule
$U_B/\si{\volt}$ & N\\
\midrule
10&	60\\
18&	45\\
26&	50\\
32,1&	44\\
40,1&	36\\
48,1&	47\\
56,2&	54\\
64,3&	84\\
72,4&	133\\
80&	200\\
88,1&	221\\
96,3&	217\\
104&	260\\
112&	202\\
120&	198\\
128&	156\\
135,9&	136\\
143,9&	110\\
151,9&	82\\
160,3&	59\\
169,3&	49\\
\bottomrule
\end{tabular}
\label{tab:tl}
\end{table}

In Tab. \ref{tab:tl} sind die gemessenen Zahlen in Abhängigkeit vom Magnetfeld eingetragen. 