\subsection{Bestimmung der Offsetspannung}
Aus den gemessenen Spannungen bei ausgeschaltetem Magnetstrom lässt sich die Offsetspannung über
\begin{align*}
  \delta U=\frac{1}{2}\sum_i (U_++U_-) \pm \frac{\Delta U}{\sqrt{2}}
\end{align*}
berechnen. Für jede der fünf Messungen wird $\delta U_i$ berechnet (siehe Tabelle \ref{tab:offsetspannung}). Der Mittelwert der Offsetspannung und der dazugehörige statistische Fehler berechnen sich zu $\overline{\delta U}=7,03$ und $\Delta \overline{\delta U}_\mathrm{stat}=0,06$. Kombiniert mit dem systematischen Fehler für jede Messung folgt 
\begin{align*}
  \overline{\delta U}=7,03 \pm 0,09.
\end{align*} 
Von allen gemessenen Spannungen wird dieser Wert abgezogen.

\subsection{Kalibration des Spektrometers}
An das aufgenommene Emissionspektrum von Barium in der Feinmessung wird eine Überlagerung aus zwei Gaußkurven mit zusätzlichem Offset
\begin{align}
  N(U)=A_0+A_1\exp \left(-\frac{(U-\mu_1)^2}{\sigma_1^2}\right) +A_2\exp\left(-\frac{(U-\mu_2)^2}{\sigma_2^2} \right)
\label{eq:gauss}
\end{align}
angepasst. Das Ergebnis ist
\begin{align*}
  A_0&=11\pm 2\\
  A_1&=157\pm 11\\
  A_2&=55\pm 6\\
  \mu_1&=150,89 \pm 0,05\\
  \mu_2&=154,9 \pm 0,1\\
  \sigma_1&=0,9 \pm 0,1\\
  \sigma_2&=0,7 \pm 0,2.
\end{align*}
Die graphische Auftragung ist in Abbildung \ref{fig:ba_fein} zu sehen.
\begin{figure}[h]
  \centering
  \includegraphics[width=0.7\textwidth]{data/Ba_fein.eps}
  \caption{Aufgezeichnetes Emissionsspektrum von Barium in Feinmessung}
  \label{fig:ba_fein}
\end{figure}

Die erste Emissionslinie entspricht der K-Linie, die zweite entspricht einer Überlagerung von drei L-Linien. Die (mittlere) Bindungsenergie von Elektronen aus der K- und L-Schale sind
\begin{align*}
  \epsilon_K&=0,07327020982\\
  \epsilon_L&=0,01099904421.
\end{align*} 
Der beobachtete $\beta^-$-Übergang hat eine Energie von
\begin{align*}
  \delta\epsilon=1,29483633.
\end{align*}
Der Impuls der Elektronen kann somit über 
\begin{align*}
  \eta=\sqrt{(\delta\epsilon-\epsilon+1)^2-1}
\end{align*}
 berechnet werden. Es folgt
\begin{align*}
  \eta_K&=1,983773179\\
  \eta_L&=2,053268796.
\end{align*}
Unter Hinzunahme des Datenpunktes $U=0 \pm 0,1$, $\eta=0$ wird die Kalibrationsgerade 
\begin{align*}
  \eta(U)=aU+b
\end{align*}
angepasst. Die Parameter sind
\begin{align*}
  a&=0,0132 \pm 0,0001\\
  b&=0,00 \pm 0,01.
\end{align*}
Die Funktion sowie die Daten sind in Abbildung \ref{fig:kal} zu sehen.
\begin{figure}[h]
  \centering
  \includegraphics[width=0.7\textwidth]{data/kal.eps}
  \caption{Kalibrationskurve für Zusammenhang von Hallspannung und Elektronenimpuls}
  \label{fig:kal}
\end{figure}

\subsection{Bestimmung des Auflösungsvermögens}
Das Auflösungsvermögen wird für beide Einstellungen (4\% und 1\% Transmission) und beide Emissionlinien bestimmt. Für die Grobmessung wird wieder die Funktion aus Gleichung (\ref{eq:gauss}) angepasst. Die Parameter sind
\begin{align*}
  A_0&=5\pm 3\\
  A_1&=175\pm 15\\
  A_2&=79\pm 12\\
  \mu_1&=151,3 \pm 0,1\\
  \mu_2&=155,2 \pm 0,2\\
  \sigma_1&=0,38 \pm 0,06\\
  \sigma_2&=0,6 \pm 0,2.
\end{align*}
Die Funktion ist in Abbildung \ref{fig:ba_grob} zu sehen.
\begin{figure}[h]
  \centering
  \includegraphics[width=0.7\textwidth]{data/Ba_grob.eps}
  \caption{Aufgezeichnetes Emissionsspektrum von Barium in Grobmessung}
  \label{fig:ba_grob}
\end{figure}

Das Auflösungsvermögen ist definiert als
\begin{align*}
  R=\frac{\delta\eta}{\eta}=\frac{\delta U}{U},
\end{align*}
wobei $\delta\eta$ für den Abstand zu $\eta$ steht, bei dem die aufgenommene Zahl der Teilchen sich halbiert. Da Impulse und Spannung zueinander proportional sind ist keine Umrechnung von Impulsen und Spannungen notwendig. Mit den angepassten Gaußkurven gilt also 
\begin{align*}
  R=\frac{\sqrt{\log(2)} \sigma}{\mu}.
\end{align*}
Die berechneten Auflösungsvermögen sind in Tabelle \ref{tab:aufloesung} zu sehen.

\begin{table}[h]
    \centering
    \caption{Auflösungsvermögen von Spektrometer}
    \label{tab:aufloesung}
    \begin{tabular}{c | c c}
      \toprule
       $R/ \%$ & 1\% Transmission & 4\% Transmission \\
      \midrule
      K-Linie & $6,43 \pm 0,06$ & $4,17 \pm 0,03$\\
      L-Linie & $5,6 \pm 0,1$ & $5,2 \pm 0,1$\\
      \bottomrule
    \end{tabular}
  \end{table}

