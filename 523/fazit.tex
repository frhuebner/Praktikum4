In der Kalibrationsmessung des Spektrums von Barium konnten bei beiden Transmissionsgraden jeweils die K- und L-Linie gut aufgenommen werden. Durch den Vergleich mit den Literaturwerten für die Übergangsenergie wurde der funktionale Zusammenhang von der Hallspannung und dem Elektronen- bzw. Positronenimpuls hergestellt. Über diesen Zusammenhang und die Ergebnisse der Messung des Emissionsspektrums von Natrium und Thallium konnte für beide Stoffe ein Kurieplot angefertigt werden. Aus dem Achsenabschnitt wurden die Übergangsenergien der beiden Stoffe berechnet. Der berechnete Wert von $\si{(0,51 \pm 0,05)\mega\eV}$ für Natrium stimmt innerhalb der Unsicherheit mit dem Literaturwert von $\SI{0,545}{\mega\eV}$\cite{naenergy} überein. Für Thallium liegt der Literaturwert von $\SI{0,76}{\mega\eV}$\cite{tlenergy} etwas außerhalb der Fehlergrenzen des ermittelten Wertes von $\si{(0,67 \pm 0,05)\mega\eV}$. Eine mögliche Erklärung wäre, dass die Untergrundmessung für die beiden Stoffe einzeln hätte gemacht werden müssen. 
