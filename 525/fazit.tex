In dem Versuch wurde die Halbwertszeit des $5/2^+$-Niveaus von $^{133}$Cs bestimmt. Dazu wurde zuerst eine Energiekalibration mithilfe der \SI{511}{keV}-Linie von $^{22}$Na und der Linien von $^{133}$Ba durchgeführt. Zudem konnte die Energieauflösung der beiden Detektoren bestimmt werden.\\

Mit der Promptkurve konnte eine Zeiteichung durchgeführt werden und die Zeitauflösung der Experimentieraufbaus bestimmt werden.\\

Die berechnete Halbwertszeit des $5/2^+$-Niveaus beträgt $T_{1/2} = (5,81 \pm 0,03) \si{\nano\second}$. Das Ergebnis kommt dem Literaturwert von \SI{6,28}{\nano\second} \cite{cs133} nahe, liegt aber viele Standartabweichungen entfernt. Die Ursache dieser Abweichung wurde diskutiert.  
