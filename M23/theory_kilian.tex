\subsection{Magnetoresistance}
The electrical properties of metals change when an external magnetic field $\vec{B}$ is applied. It should be noted that we only consider the transverse magneto resistance, although there will also be a longitudinal magnetoresistance. In a simple model for the transverse magnetoresistance, the metal consists of two charge carriers, e.\ g.\ electrons and holes. Both types contribute to the total current
\begin{align*}
    \vec{j}_\mathrm{tot}=\vec{j}_1+\vec{j}_2.
\end{align*}
Both charge carriers have a corresponding resistivity inversely proportional to the conductivity
\begin{align*}
    \rho_i=\frac{1}{\sigma_i}.
\end{align*}
The total resistivity is then given by 
\begin{align*}
    \rho=\frac{1}{\sigma_1+\sigma_2}.
\end{align*}
For the dependence of the external magnetic field we are interested in the change of the resistivity
\begin{align*}
    \Delta \rho=\rho-\rho_0,
\end{align*}
where $\rho_0$ is the zero field ($B=0$) resistivity. According to Kohlers Rule one then has the functional relationship
\begin{align}
    \frac{\Delta \rho}{\rho_0}=F\left( \frac{B}{\rho_0} \right)
    \label{eq:kohler_1}
\end{align}
for some function $F$ which only depends on the metal.\\ 

To measure the resistivity of copper the 4-point method is used in order to not measure the dominating rersistance of the wires and contacts. The resistivity can then be calculated by
\begin{align*}
    \rho=\frac{AU_\mathrm{R}}{I_\mathrm{R}l_\mathrm{R}},
\end{align*}
where $A$ is the cross-sectional area of the sample, $U_\mathrm{R}$ is the voltage measured over the distance $l_\mathrm{R}$ and $I_\mathrm{R}$ is the applied current.

\subsection{Thermal Conductivity}
The thermal conductivity of a metal also changes when an external magnetic field is applied. This can be understood by considering the deflection of the electrons due to the Lorentz force. Equivalent to \ref{eq:kohler_1} Kohler proposed that for the thermal conductivity
\begin{align*}
    \frac{\Delta \kappa}{\kappa}=G \left( \frac{B\kappa_0}{T} \right)
\end{align*}
holds for some function $G$ depending on the metal only.